\documentclass[a4paper,18pt]{jsarticle}
\input{/Users/yamadaritsuki/TeX/setting.tex}

\begin{document}
\begin{tcolorbox}
  {\Large \bf Real and Complex Analysis Ch1. レジュメ} \hfill  \today
\end{tcolorbox}

\section{発表の項立て}
\subsection{リーマン積分の拡張としてのintroduction}
\begin{itemize}
  \item リーマン積分の話
  \item σ-加法族, 測度, 単関数が必要だよねの話
\end{itemize}

\subsection{舞台となる集合 : σ-加法族とボレル集合族}
\begin{itemize}
  \item 位相空間と可測空間の対応[2, 3]
  \begin{itemize}
    \item 位相空間に関するコメント[4, 5]
    \item 可測空間に関するコメント[6]
    \item $[0, \infty]$上の位相と代数 [4, 22]
  \end{itemize}
  \item 可測空間の例 [12(a)]
  \item ボレル集合族
  \begin{itemize}
    \item 位相空間は可測空間か?[11]
    \item ボレル集合族の定義[10, 11]
  \end{itemize}
  \item 位相空間, 可測空間, ボレル集合族
  \item 言葉の使い方 [21]
\end{itemize}


\subsection{集合の測り方 : 測度}
\begin{itemize}
  \item 定義 [18]
  \item 性質 [19]
  \item 例 [20]
  \item (確率との関係性)
\end{itemize}

\subsection{舞台となる関数 : 可測関数と単関数}
\subsubsection{可測関数}
\begin{itemize}
  \item 可測関数の性質たち
  \begin{itemize}
    \item 定理1.7 と 定理1.12(d) [7, 12(b), (d)]
    \item 可測関数の判定(定理1.12(c)) [12(c)]
  \end{itemize}
  \item 可測関数の例
  \begin{itemize}
    \item 複素関数, 絶対値, 実部/虚部, 和, 積の可測性 [8, 9(a)~(c)]
    \item 特性関数[9(d)]
    \item 位相みたいな [9(e)]
  \end{itemize}
  \item 可測関数について
  \begin{itemize}
    \item 上限, 上極限, 各点収束極限の可測性 [13, 14]
    \item 正成分と負成分の可測性 [15]
  \end{itemize}
\end{itemize}

\subsubsection{単関数}
\begin{itemize}
  \item 単関数の定義 [16]
  \item 可測関数に収束する単関数の存在 [17]
\end{itemize}

\subsection{ルベーグ積分 (非負関数)}
\begin{itemize}
  \item ルベーグ積分の定義 [23]
  \item ルベーグ積分の自明な性質 [24]
  \item 2つの線形性 [25]
  \item ルベーグの単調収束定理 [26]
  \item 線形性の一般的な証明 (その系としての無限級数との交換) [27]
  \item Fatouの補題 [28]
  \item 変数変換 [29]
\end{itemize}

\subsection{ルベーグ積分 (複素関数)}
\begin{itemize}
  \item $L^1(\mu)$の定義 [30]
  \item 複素関数のルベーグ積分の定義 [31]
  \item 線形性 [32]
  \item 絶対値のルベーグ積分 [33]
  \item ルベーグの優収束定理 [34]
\end{itemize}

\subsection{測度0 : ほとんど至る所で}
\begin{itemize}
  \item 測度0はゴミだけど大切 [35]
  \item 測度の完備化 [36]
  \item 測度0を考慮した可測性のお話 [37]
  \item ほとんど至る所で成立する定理たち
  \begin{itemize}
    \item ほとんど至る所での無限級数の収束とルベーグ積分との交換 [38]
    \item ほとんど至る所での関数の一致 [39]
    \item 平均に関する定理 [40]
    \item 測度の無限級数和 [41]
  \end{itemize}
\end{itemize}


\end{document}
